\section{Registrering}

Billeder taget på MRI/PET og PET/CT skannere kan som regelt ikke processeres sammen
grundet flere faktorer. Patienten ligger sjældent præcis på samme måde, billederne
bliver taget i forskellige opløsninger og patienten kan have implantater der
forvrænger billederne. For at klargøre billederne skal de derfor co-registreres.

Ved co-registrering forsøger man at få alle billederne til at ligge i samme rum.
I forhold til MRI billederne er co-registrering ofte trivielt. At co-registrere
et MRI og CT billede er derimod vanskeligere. Derfor har vi valgt to forskellige
metoder.

I samme omgang som vi co-registrere billederne er vi interesserede også at finde
en maske. Masken skal bruges til at begrænse udregningen af sCT billedet så vi
ikke bruger lang tid på at lede efter knogle i luften rundt om patienten.

\subsection{Co-registrering af UTE og T1 billeder}

Til co-registrering af UTE og T1 billederne har vi valgt at bruge Insight ToolKit
(ITK). Herfra benytter vi en implementation af Mattes Mutual Information algoritme
samt linær translation og interpolering.

\subsection{Co-registrering af UTE og CT}

Co-registrering af UTE og CT billeder er modsat UTE/T1 en ikke-triviel opgave. Vi
har valgt en landmark baseret løsning fra MINC's toolkit

\subsection{Generering af maske}

Ligesom ved co-registrering bruge vi ITK for at udregne en maske. I første step
bruger vi en implementering af Otsu thresholding algoritme for at finde en binær
repræsentation. For at sikre vi ikke misser noget af kanterne udvider vi det 
binære billede vha. en neighborhood-connected algoritme med 2-3 pixel i x, y og z
retningerne. Til sidst inverteres billedet og vi har dermed en binær maske vi kan
bruge til at begrænse området sCT algoritmen arbejder på.