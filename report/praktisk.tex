\section{Praktisk}

\subsection{Registrering}

\todochristoffer{Der mangler referencer til registrering}

\subsubsection{Co-registrering af UTE og T1 billeder}

Til co-registrering af UTE og T1 billederne har vi valgt at bruge Insight
ToolKit (ITK). Herfra benytter vi en implementation af Mattes Mutual
Information algoritme samt linær translation og interpolering.

\subsubsection{Co-registrering af UTE og CT}

Co-registrering af UTE og CT billeder er modsat UTE/T1 en ikke-triviel
opgave. Vi har valgt en landmark baseret løsning fra MINC's toolkit

\subsubsection{Generering af maske}

Ligesom ved co-registrering bruge vi ITK for at udregne en maske. I første step
bruger vi en implementering af Otsu thresholding algoritme for at finde en binær
repræsentation. For at sikre vi ikke misser noget af kanterne udvider vi det 
binære billede vha. en neighborhood-connected algoritme med 2-3 pixel i x, y og z
retningerne. Til sidst inverteres billedet og vi har dermed en binær maske vi kan
bruge til at begrænse området sCT algoritmen arbejder på.

\subsection{Metoden}

\todochristoffer{Praktisk omkring metoden er ikke beskrevet}

\subsection{Analyse af implementeringen}

\todochristoffer{Analysen af implementeringen er ikke beskrevet}
