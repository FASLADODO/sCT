\section{Praktisk}

\subsection{Beregning af modellen}

Vi har valgt at implementerer metoden i Matlab da det giver os mulighed
for hurtig prototyping. Derudover indeholder Matlab også en standard
implementering af både k-means og EM algoritmerne.

For at beregne modellen skal vi først indlæse MR og CT billederne
samt tilhørende maske for hver patient. Til det formål har vi brugt
nii\_load fra matlab toolboxen "Tools for NIfTI and ANALYZE
image\footnote{\url{http://www.mathworks.com/matlabcentral/fileexchange/8797-tools-for-nif ti-and-analyze-image}}".
Herefter løber vi billederne igennem og udtager de værdier der ligger
inden for masken. Alle værdier lægges i én 16-dimensionel liste.

For at udregne k-means bruger vi den indbyggede k-means algoritme, samt
en sammensat liste med de patienter, som vi ønsker at træne på
modellen.

Output fra k-means algoritmen gives til EM algoritmen, samt samme liste
med patienter. Output herfra er vores endelige model parametre.

\subsection{Beregning af sCT}

Selve udregningen af sCT implementeres på samme måde, som beskrevet i
teorien. Koden er en mindre overskuelig ud, men det er primært fordi
vi var nødt til at omstrukturere flere arrays, da de påvirkede
køretiden negativt i en ekstrem grad. Vi udregner også de konstante dele
af algoritmen på forhånd, igen for at mindske køretiden.

Fordi beregningen er voxel baseret er det muligt for os at udregne flere
punkter parallelt. Det tager os omtrent to minutter at udregne et sCT
hvilket er samme køretid Johannson et al. nævner.

Resultatet af algoritmen gemmer vi vha. nii\_save, det er ikke
nødvendigt for os at gemme headeren, eftersom vi i rekonstruktionen
alligevel erstatter den med headeren fra tilhørende $\mu$-map.

\subsection{Transformering til \texorpdfstring{$\mu$}{u}-map} 

Når vi har vores sCT, skal vi have lavet det om til et $\mu$-map
på Dicom format. For at opnå dette konverterer vi det først til
Minc. I Minc lægger vi 2048 til for at gøre op for en justeringen i
værdierne, der opstår i konveteringen fra Minc til Nifti. Derefter
resampler vi billederne til samme format, som UTE $\mu$-mappet fra den
tilhørende scanning og erstatter header information i sCT'et med det fra
$\mu$-mappet, så det kan benyttes, som et $\mu$-map på scanneren. Til
sidst konverterer vi det videre til Dicom, hvilket er det format scanneren
benytter.








