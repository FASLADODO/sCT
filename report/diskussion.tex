\section{Diskussion}
\subsection{Kvalitet}

\todolasse{Kvaliteten af vores implementering er ikke beskrevet}

\subsection{Problemer fx. Artifakter}

\todolasse{Har ikke skrevet om problemer i forhold til sCT}

\subsection{Over tid}

\todolasse{Har ikke beskrevet korrektheden over tid}

\subsection{Billedformater}

I løbet af projektet har vi oplevet en del problemer med konverteringer
mellem filtyper. Vi benyttertre forskellige. Siemens scannerne producerer
Dicom billeder som vi arbejder med i Osirix, der er et medicinsk
billedbehandlisk program. Men det er et meget stort format, der ikke er
let at arbejde med. 

På Rigshospitalet er normen at konvertere til Minc, men vi havde planer,
om at bruge ITK, og Nifti var bedre understøttet i det. I sidste ende
endte vi med at registrere vores CT billeder i Minc, så vores billeder gik
igennem alle tre typer. 

Fra Dicom til Minc mister man en skalar, som man skal tage højde for, når
man konvertere tilbage. Ved konvertering fra Minc til Nifti, hvilket vi
kun gør med vores registrerede CT billeder, bliver de laveste værdier til
-1024, dem trækker vi endnu 1025 fra, så de kommer til at matche vores
laveste værdier.

\todo{Uddyb tab ved Dicom til Minc}

At konvertere til Dicom er ikke helt trivielt, så vi konverterer først til
Minc, hvor vi lægger 2048 til for at gøre op for det offset, der skete ved
den tidligere konvertering. 

Mange af disse problemer har vi opdaget så sent i processen, at vi ikke
har kunnet nå at tage hånd om dem. Den gennemsnitlige værdi i alle vores
Nifti billeder, som vi har trænet og testet på, har været for lav, og at
lægge en værdi til ved tilbagekonverteringen udbedrer det ikke, men ændrer
bare værdierne til samme størrelse, som i det målte CT.



