\section{Introduktion}

\todochristoffer{Læs introduktion}

Til diagnosticering af patienter med visse former for sygdomme i hjerneregionen, fortages der ofte scanninger vha. Positron Emission Tomography (PET). 

PET scannere fungerer ved, at man injekterer et radioaktivt sporstof
ind i et legeme. Dette sporstof flyder ud i blodet og bliver bundet
der, hvor der benyttes mest energi, for eksempel en kræftcelle i
hjernen. Dette udsender stråling, som PET scanneren kan opfange.
Strålingsaktiviteten måles for at finde ud af, hvor i legemet der er
mest aktivitet.\\

\todo{PET-billede, CT billede og MR billede. Vi må lige få styr på formateringen senere}

PET har dog visse problemer. Man fanger kun strålingskoncentration,
så det er umuligt at se knogle, luft eller blødt væv. Dette giver
problemer, da forskelligt væv afbøjer photoner i forskellige grad,
og luft næsten ikke gør, forårsager det forringelser i kvaliteten
af scanningen, da man ikke er sikker på hvad photoner har bevæget
sig igenn, og dermed, hvor meget de er blevet afbøjet. Hvis man ikke
kender dette, kan man ikke med sikkerhed bestemme deres oprindelsessted.
Dette betragtes, som den væsentligste årsag til foringelse af
billederne.\cite{vigtighedAfAttenuation} At udbedre denne forringelse er
attenuationskorrigering.

For at udbedre denne forringelse har man kombineret Computed
Tomography- (CT) og PET scannere. CT er et røngtenstrålingsscan,
der registrerer placering af knogle. Denne information kan man bruge
til at attenuationskorrigere og dermed få et mere korrekt resultat.
CT har dog andre problemer, da den ikke indeholder information om de
bløde vævstyper i hjernen. Derudover har røntgenstråling også
en væsentligt risiko for at forårsage yderlig skade. \cite{skadeligCT}

For at registrerer det bløde væv kan Magnetic Resonance(MR) benyttes.
MR magnetiserer et legeme svagt, når dette stoppes vil legemet hurtigt
gendanne sit oprindelige magnetfelt. Hastigheden med hvilken den gendannes
afhænger af vævet. MR scanneren varierer magnetiseringen og starter og
stopper det gentagne gange for at kortlægge legemet. Dette giver billeder
af hjernevæv i høj kvalitet.

Som med CT er MR også blevet sammensat med PET til en PET/MR scanner, men
den kan ikke bruges til attenuationkorrektion, da knogle og luft begge 
gendanner deres magnetfelt så hurtigt, at det er meget svært at måle.
Effekten er at knoglen kommer til at ligne luft på billederne. Dette
problem er forsøgt løst på flere måde.

På Rigshospitalet har man hidtil kombineret MR billeder med CT
billeder for at kunne lave attenuationskorrigerede billeder kaldet FullCT.
Dette medfører dog det tidligere nævnte problem med den skadende
effekt af røngtenstråling og man skal flytte patienten rundt mellem
flere scannere. Da scanningerne er foretaget på forskellige scannere
kan man ikke forvente at patienterne ligger ens. Dette problem skal
også løses. Derfor er man særdeles interesserede i, at foretage
attenuationskorrigering alene vha. MR. \\

Der er to hovedretninger til at udregne CT data ud fra MR data. De
anatomiske og de voxelbaserede. De anatomiske forsøger at kortlægge
hjernen og beskrive generelle træk, de matcher patient data op imod et
udregnet atlas, og forsøger at korrigere udfra det. Det giver nogle
problemer ved atypisk anatomi, men har vist sig at producere gode
resultater.

Alternativt er der voxelbaserede metoder, som for hver voxel forsøger
at bestemme hvilken type væv den tilhører, men her er det selvfølgelig
virkelig et problem, at man ikke kan skelne mellem luft og knogle.

Ved at benytte meget korte sekvenser, kaldet UTE (Ultrashort Echo Times,)
er det muligt at registrere knogle på MR-scannere. Man måler fra to
forskellige vinkler og med to forskellige echo tider. Der hvor der er
sket en markant ændring i mellem de to billeder, kan da formodes at
være knogle. Implementation af dette på Siemens scannerne, som
Rigshospitalet benytter, har dog vist sig ikke at være på højde med
kombinationen af CT og MR billeder.

Målet med denne opgave er at implementere metoden beskrevet i
Johansson et al., som kan udregne et substitut-CT (sCT) ud fra fire
UTE sekvenser og et T2-vægtet MR billede for en vilkårlig patient. Dette
vil vi gøre ved at træne en Gaussian Mixture Regression model ud fra
disse serier samt et CT målt på samme patient.
