\section{Introduktion}
\subsection{Hvad er der skrevet om området}

\todolasse{Mangler at beskrive hvad der er sket på området}

\subsection{Hvad er målet med vores opgave}

\todochristoffer{Læs}

Når en patient får foretaget en PET/MR skanning indeholder den kun
information om de vandholdige legemer i kroppen. Derfor tager man også en
CT for at attenuationskorrigere MR billederne. Der er flere problemer ved denne
fremgangsmåde. For det første er at patienten skal flyttes rundt mellem flere
skannere, hvilket koster resourcer og tid, derudover kan sammensætningen af
de to skanninger, forårsage forringelser af billedkvaliteten og CT scanninger
kan også være kræftfremkaldende.

Et forskningshold i Umeå har fundet en metode til at beregne et substitute-CT
(sCT) fra PET/MR skanningen som kan erstatte CT skanningen. Metoden benytter
5 MRI sekvenser; fire UTE og en T2. Da vi ikke har T2 på alle patienter,
benytter vi T1-vægtede sekvenser i stedet. Ved hjælp af MR sekvenserne og
et CT billede trænes en parametriseret Gaussian mixture regression model, som
kan bruges til at udregne et sCT for andre patienter.\\

Vores håb er at vi kan implementere samme metode til brug for Rigshospitalet.
Vi vil desvidere undersøge korrektheden og kvaliteten af vores sCT for at se
om det kan bruges klinisk.\\



