\section{Introduktion}

\todochristoffer{Læs introduktion}

\todolasse{Mere fysik? Mindre Fysik? Photoner? Positroner? Elektroner?
læs op på kilde 12 og 13. Mere atlas beskrivelse eller måske skal det bare i
diskussionen}

\todo{Læs Zaidi, H., Sossi, V., 2004. Correction for image degrading factors
is essential for accurate quantification of brain function using PET. Med.
Phys. 31, 423 – 426.}

Til diagnosticering af patienter fortages der ofte scanninger af patienter på
Positron emission tomography (PET) scannere i forbindelse med sygomme og
lignede i hjernen. PET scannere fungerer ved, at man injektere et radioaktivt
sporstof ind i et legeme. Dette sporstof vil da binde sig til f.eks.
kræftceller i hjernen. Dette udsender stråling, som PET scanneren kan opfange
den og udregne placeringen af sporstoffet.\\

<PET-billede, CT billede og MR billede. Vi må lige få styr på
formateringen senere>\\

PET har dog visse problemer. Man fanger kun strålingskoncentration, så det er
umuligt at se knogle. Det giver problemer, da kraniet afbøjer photonerne giver
det en forringelse i kvaliteten af scanningen, da man ikke er sikker på dens
oprindelsessted. Dette betragtes, som den væsentligste årsag til foringelse af
billederne (evt. kilde: Zaidi, H., Sossi, V., 2004). At udbedre denne
forringelse er attenuationskorrigering. 

For at udbedre denne forringelse har man kombineret Computed Tomography- (CT)
og PET scannere. CT er en røngtenstrålingsscan, der viser en tydeligt knogle,
og denne kan man bruge til at attenuationskorigere og få et mere korrekt
resultat. CT har dog andre problemer. Den viser meget lidt indhold i hjernen og
kan desuden være yderlig skadelig for hjernen. \\

Til at visse vævet i hjerne benyttes Magnetic Resonance Imaging (MRI). MR
magnetiserer et legeme svagt, når dette stoppes vil legemet hurtigt gendanne
sit oprindelige magnetfelt. Hastigheden med hvilken den gendannes afhænger
af vævet. MR scanneren varierer magnetiseringen og starter og stopper det
gentagne gange for at kortlægge legemet. Dette giver billeder af hjerne i
høj kvalitet. Dette er også blevet sammensat med PET til en PET/MR scanner,
men den kan ikke bruges til attenuationkorrektion, da knogle gendanner sig
magnetfelt så hurtigt, at det ikke er til at måle. Så knogle ligner luft på
billederne. Til at løse dette har man ofte også lavet et CT scan og benyttet
dette til attenuationskorrigering. Dette medfører dog det tidligere nævnte
problem med den skadende effekt af røngtenstråling, man skal flytte patienten
rundt mellem flere scannere og da scanningerne er foretaget på forskellige
scannere kan man ikke forvente at patienterne ligger ens. Dette problem skal
også løse. Derfor er det meget interessant, om man kan foretage
attenuationskorrigering udfra MR. \\

Der er to hovedretninger til at udregne CT data ud fra MR data. De anatomiske
og de voxelbaserede. De anatomiske forsøger at kortlægge hjernen og beskrive
generelly træk, de matcher patient data op imod atlases og forsøger at
korrigere udfra det. Det giver nogle problemer ved atypisk anatomi, men har
vist sig at producere gode resultater (kilde kilde kilde). Alternativt er der
voxelbaserede metoder, som for hver voxel forsøger at bestemme hvilken type
væv den tilhører, men her er det selvfølgelig virkelig et problem, at man ikke
kan skelne mellem luft og knogle

Ved at benytte meget korte sekvenser UTE (Ultrashort Echo Times) er
det muligt, at se knogle. Man måler fra to forskellige vinkler og med to
forskellige echo tider. Det der er forsvundet i mellem de to billeder er
da forventligt knogle. Der er stadig nogle problemer med, at man ikke med
sikkerhed kan se forskel på knogle og blødt væv nær luft.

Målet med denne opgave er at implementere en metode beskrevet i
(JOHANSSON), som kan udregne et substitut CT udfra de fire UTE sekvenser
og et T2-vægtet MR billede for en vilkårlig patient. Dette vil vi gøre ved
at træne en Gaussian Mixture Regression model ud fra de samme serier samt et
målt CT på samme patient. Til forskel fra dem, vil vi benytte et T1 vægtet
billede til at træne algoritmen, da der ikke er lavet T2 på alle
rigshospitalets hjerner.
