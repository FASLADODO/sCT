\section{Forord}

Dette er vores bachelorprojekt, hvilket vi har udarbejdet i perioden 3.
februar til 16. juni 2014. Vi har skrevet det hos Datalogisk Institut
ved Københavns Universitet i samarbejde med Rigshospitalet. Vores
hovedvejleder var Adjunkt Sune Darkner med ekstern vejleder ph.d,
PET/MR-fysiker Adam Espe Hansen.

Vi startede dette projekt med faglige kompetencer på linje med, hvad
man kan forvente af en datalogistuderende på tredje år. Det vil sige at vi
ingen reel erfaring havde med billedbehandling og maskinlæring. På begge
felter har vi i løbet af projektet tilegnet os viden om metoder og
teknikker, som vi sidenhen har benyttet.

Vi har i løbet af projektet opdaget, at nogle af vores indledende
beslutninger ikke har været optimale, og nogle af dem har vi, grundet
projektets tidsramme, ikke været i stand til at udbedre.

Det forventes at læseren kender til basale metoder og termer inden for
billedbehandling og maskinlæring.

Vi vil gerne takke pd.d Flemming Andersen, Rigshospitalet, for den
oprindelige ide til projektet og løbende vejledning. Vi vil gerne
takke MSc Claes Ladefoged, Rigshospitalet, for stor assistance under
projektet, og flere af de scripts, vi benyttede, er lavet af ham. Tak skal
det også lyde til Overlæge Ian Law, Rigshospitalet, for at tage sig tid
til at se på vores resultater og give feedback. Endelig skal der lyde
tak til Lektor Francois Bernard Lauze, DIKU, for hjælp med udformning af
projektet og indledende vejledning.


