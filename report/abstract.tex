\abstract

\textbf{Formål:} Undersøg brugbarheden af sCT til rekonstruktion af PET og vurdér kvaliteten. Undersøg desuden om sCT er stabilt over tid. \textbf{Resultat:} Flere regioner i hjernen på PET baseret på sCT har forskelle på mere end 10\% i forhold til PET baseret på CT. Værdierne på sCT billedet ser ud til at falde med tiden. \textbf{Konklusion:} Kvaliteten af sCT ser ikke ud til at være god nok til generelt brug. Det er muligt den kan bruges ved diagnosticering af Alzheimers. Værre endnu, så ser det ud til at metoden ikke er stabil. Periodiske beregninger af modeller ser derfor ud til at være nødvendige.\\

\textbf{Purpose:} Investigate the usefulness of sCT in PET reconstruction and evaluate the quality. Further investigate if the sCT is stable with time. \textbf{Result:} Several regions of the brain in the PET image based on sCT shows differences larger than 10\% difference compared to PET image based on CT. \textbf{Conclusion:} The quality of sCT seems to be too low for general use. It is possible the sCT could be used in diagnosing Alzheimer's disease. Even worse, the method proves not to be stable over time. Thus periodic calculation of the model seems to be necessary.
