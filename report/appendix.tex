\section{Appendix}

\subsection{Synopsis}

\subsubsection{Problemformulering}
Konstruer et sCT med metoden, som beskrevet i "CT substitute derived from 
MRI sequences with ultrashort echo time" af Adam Johansson m. fl. fra 2011~\cite{johansson}.
Er sCT billedet sammenligneligt med et CT, og er metoden stabil over tid?

\subsubsection{Begrundelse}
Når en patient får foretaget en PET/MR skanning indeholder den kun
information om de vandholdige legemer i kroppen. Derfor tager man også en
CT skanning for at kunne se patientens knoglestruktur. Der er to problemer
ved denne fremgangsmåde. Den første er at patienten skal flyttes rundt
mellem flere skannere, hvilket koster resourcer og tid. Det andet problem
er sammensætningen af de to skanninger, der kan forårsage forringelse af
billedkvaliteten.

Et forskningshold i Umeå har fundet en metode til at beregne et
substitute-CT (sCT) fra PET/MR skanningen som kan erstatte CT skanningen.
Metoden benytter tre MRI sekvenser (to UTE og en T2). Ved hjælp af MR sekvenserne
og et CT billede trænes en parametriseret Gaussian mixture regression model.\\

Vores håb er at vi kan implementere samme metode til brug for Rigshospitalet.
Vi vil desvidere undersøge korrektheden og kvaliteten af vores sCT for at se
om det kan bruges klinisk.\\

Vi vil i vores projekt gøre følgende.
\begin{itemize}
    \item Redegøre for attenuation correction.
    \item Beskrive dannelse af PET, MR, CT og UTE billeder.
    \item Redegørelse for dannelse af sCT med udgangspunkt i Johansson2011.
    \item At kunne fremstille substitute-CT (sCT) udfra PET/MR- og
        UTE-sekvenser,
        som beskrevet i ~"Johanson 2011". Og ud fra dette danne vores eget 
        sCT-uMap.
    \item Sammenlign vores sCT med det faktiske CT og Umeås sCT.
    \item At bedømme kvaliteten af vores implementation. 
Scanneren kan drifte og modificeres over tid, hvilket kan ændre på
scanningsresultaterne. Hvilken effekt har dette på vores implementation.
    \item Analysere korrektheden af en implementation og redegøre for
        eventuelle mangler og svagheder.
\end{itemize}

\subsubsection{Arbejdsplan}
\begin{enumerate}
    \item Litteraturstudie 
    \item Implementere metode
    \item Teste og analysere implementationen
    \item Rapportskrivning
\end{enumerate}

\begin{gantt}{5}{16}
    \begin{ganttitle}
      \numtitle{1}{1}{16}{1}
    \end{ganttitle}
    \ganttbar{(1)}{0}{9}
    \ganttbar{(2)}{1}{11}
    \ganttbar{(3)}{5}{9}
    \ganttbar{(4)}{4}{12}
\end{gantt}

\newpage

\subsection{Kildekode}

Kildekoden til vores projekt kan findes på \url{https://github.com/Wadum/sCT/tree/master/src}.