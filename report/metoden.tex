\section{Metoden}
\subsection{Gaussian Mixture Model}

\subsubsection{At finde modellen}
Vi bruger en gaussian mixture til at beregne voxelværdier
for et sCT. Til hver patient har vi fem MR-billeder og et CT-billede.

For at øge præcisionen udregner vi for hvert MR-billede
to nye billeder. De nye billeder beregnes ved at se på en 3x3x3
kube omkring hvert voxel og finde henholdsvis middelværdi og varians.

Herefter har vi 15 MR-billeder og et CT-billede, som vi vil klassificere
med en gaussian mixture. Fra Johansson et al. ved vi, at vi kan få
gode resultater med 20 gaussians. 

For at finde parametrene til hver gaussian bruger vi
Expectation-Maximization (EM) algoritmen på en sammensætning af patienter, som
vi vil træne på. Vi starter EM algoritmen med resultatet af k-means på
data, hvor k er sat til 20. Dette betyder, at vi ikke behøver køre
EM-algoritmen flere gange, og øger sandsynligheden for et godt resultat, da
data allerede er klassificeret.

\subsubsection{Beregning af sCT}

Vi genererer sCT-værdierne ved at udregne den forventede værdi af CT baseret på de 15 MR-billeder. 

Som beskrevet i Johansson et al., så udregnes den forventede værdi af CT ved

\begin{equation}
(X_1 | X_2 = x_2) = \frac{\Sigma^{N}_{i=1} \bar{\mu}_{1,i} a_i h_i(x_2)}{\Sigma^{N}_{i=1} a_i h_i(x_2)}
\end{equation}

Hvor $X_1$ er CT-værdien, $X_2 = (X_2 \dots X_k)$ er MR-værdierne, $a_i$ er blandingsforhold og $N$ er antallet af MR-billeder samt

\begin{equation}
 \bar{\mu}_{1,i} = \mu_{1,i} + \Sigma_{1 2, i} \Sigma^{-1}_{22, i}(x_2 - \mu_{2,i})
\end{equation}
og
\begin{equation}
h_i(x_2)= \frac{1}{( 2 \pi )^{k/2}|\Sigma_{22,i}|^{1/2}}\exp^{-\frac{1}{2}(x_2 - \mu_{2,i})^T \Sigma^{-1}_{22, i}(x_2 - \mu_{2,i})}
\end{equation}

$\mu_{1,i}$, $\mu_{2,i}$, $\Sigma_{1 1, i}$, $\Sigma_{1 2, i}$, $\Sigma_{2 1, i}$, $\Sigma_{2 2, i}$ er værdierne $\mu_i$ og $\Sigma_i$ estimeret med EM algoritmen og segmenteret som

\begin{equation}
\mu_i = \left(\begin{array}{c}
\mu_{i,1} \\ 1 \times 1 \\
\mu_{i,1} \\ 1 \times (k - 1)  
\end{array}\right)
\end{equation} 
\begin{equation}
\Sigma_i = \left(\begin{array}{c c}
\Sigma_{11,i} & \Sigma_{12,i} \\ 1 \times 1  & 1 \times (k-1)\\
\Sigma_{21,i} & \Sigma_{22,i} \\ (k-1) \times 1  & (k-1) \times (k-1)\\  
\end{array}\right)
\end{equation}

\subsection{Hvordan han vi valgt at implementere den}

Til metoden benytter vi for hvert træningssæt et CT, en T2 vægtet MR-sekvens
og fire MR UTE-sekvenser. Vi benytter ITK til at udregne et
billede med middelværdi og et med varians for hvert MR-billede.

Disse henter vi ind i matlab, hvor vi benytter fitgmdist til at lave en
gaussian mixture model. Med denne kan vi udregne et sCT udfra den
samme slags MR-sekvenser, som vi brugte til at lave modellen, samt deres
gennemsnits- og variansbilleder.

