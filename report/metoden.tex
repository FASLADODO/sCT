\section{Metoden}
\subsection{Gaussian Mixture Model}

\subsubsection{At finde modellen}
Vi bruger en mixtur af gaussians til at beregne voxel værdier
for et sCT. For hver patient har vi 5 MR-billeder og 1 CT billede.

For at øge præcisionen udregner vi for hvert MR-billede
to nye billeder. De nye billeder beregnes ved at se på en 3x3x3
kube omkring hvert voxel og finder henholdsvis middelværdi og varians.

Herefter har vi 15 MR-billeder og 1 CT-billede som vi vil klassificere
med en mixtur af gaussians. Fra Johannson et al. ved vi at vi kan få
gode resultater med 20 gaussiske fordelinger. 

For at finde parametrene til hver gaussiske fordeling bruger vi
Expectation-Maximization (EM) algoritmen på en sammensætning af patienter
vi vil træne på. Vi starter EM algoritmen med resultatet af k-means på
dataen, hvor k er sat til 20. Dette betyder at vi ikke behøver køre
EM-algoritmen flere gang og øger sandsynligheden for et godt resultat, da
dataen allerede er klassificeret.


\todochristoffer{Matematiske ekslempler}
\todolasse{Gennemlæs}


\subsubsection{At udregne et sCT}

\todochristoffer{Beregning af sCT er ikke beskrevet}

\subsection{Hvordan han vi valgt at implementere den}

\todochristoffer{Læs}

Til metoden benytter vi for hvert træningssæt et CT, en T1 vægtet MR-sekvens
og 4 MR UTE-sekvenser. Vi benytter Insight Toolkit (ITK) til at udregne et
billede med middelværdi og et med varians for hvert MR billede.

Disse henter vi ind i matlab, hvor vi benytter fitgmdist til at lave en
gaussisk mixture fordelingsmodel. Med denne kan vi udregne et sCT udfra den
samme slags MR-sekvenser vi brugte til at lave modellen, samt deres
mean- og variansbilleder.
