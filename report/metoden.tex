\section{Metoden}
\subsection{Gaussian Mixture Model}

\subsubsection{At finde modellen}
Vi bruger en mixtur af gaussians til at beregne voxel værdier
for et sCT. For hver patient har vi 5 MR-billeder og 1 CT billede.

For at øge præcisionen udregner vi for hvert MR-billede
to nye billeder. De nye billeder beregnes ved at se på en 3x3x3
kube omkring hvert voxel og finder henholdsvis middelværdi og varians.

Herefter har vi 15 MR-billeder og 1 CT-billede som vi vil klassificere
med en mixtur af gaussians. Fra Johannson et al. ved vi at vi kan få
gode resultater med 20 gaussiske fordelinger.

For at finde parametrene til hver gaussiske fordeling bruge vi 
Expectation-Maximization(EM) algoritmen. Vi starter EM algoritmen med
tilfældige værdier for ikke at ramme et lokalt minimum. Herefter er
det blot at udregne afstanden til hvert voxel-sæt og klassificere dem.
Efter klassifikationen udregnes nye paramtetre for hver gaussiske
fordeling.

Vi gentager herefter samme process indtil differencen imellem parametrene
bliver små nok.
\todochristoffer{Matematiske ekslempler}
\todolasse{Gennemlæs}


\subsubsection{At udregne et sCT}

\todochristoffer{Beregning af sCT er ikke beskrevet}

\subsection{Hvorfor den ikke og en anden?}

\todolasse{Valg af metoden er ikke beskrevet}

\subsection{Hvordan han vi valgt at implementere den}

\todolasse{Implementering af metoden er ikke beskrevet - Overlap med praktisk}