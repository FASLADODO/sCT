\section{Teori}
\subsection{Kort introduktion til de forskellige type data og dannelse af disse.}

\todochristoffer{Læs}
\todolasse{Skriv om PET effekt på MR. Mere om forskel på UTE-sekvenserne}


Til metoden benyttes Magnetic resonance- (MR) og Computed
tomography-billedsekvenser (CT). CT er en røngtenscanning, som måler
evnen til at blokkere stråling, hvilket har en direkte relation til
elektrondensitet. Knoglen er meget tydelig i disse billeder. MR magnetiserer
kroppen ganske svagt, når dette stoppes vil kroppens magnetisering bruge
ganske kort tid på at gendannes, og der måles på denne for at kortlægge
struktur.

Vi benytter 5 MR-sekvenser: T1, som måles efter kroppen har gendannet en vis
mændge magnetisme, og derudover 4 Ultrashort echo time MR-sekvenser (UTE)
taget fra 2 forskellige vinkler og varierende ekko tider. Når man sænker echo
tiden betydeligt bliver det muligt at se knogle på MR billeder, men ikke med
en præcision, som kan matche CT.


\subsection{Registrering}

\todochristoffer{Mangler referencer i teori til registrering}

Billeder taget på PET/MR og PET/CT skannere kan som regelt ikke processeres
sammen grundet flere faktorer. Patienten ligger sjældent præcis på samme
måde, billederne bliver taget i forskellige opløsninger og patienten kan have
implantater der forvrænger billederne og de ligger også i forskellige image
spaces. For at klargøre billederne skal de derfor co-registreres.

Ved co-registrering forsøger man at få alle billederne til at ligge i samme
rum. I forhold til MRI billederne er co-registrering ofte trivielt. At
co-registrere et MRI og CT billede er derimod vanskeligere. Derfor har vi valgt
to forskellige metoder.

\todo{Mangler der noget her?}

I samme omgang som vi co-registrere billederne er vi interesserede også
at finde en maske. Masken skal bruges til at begrænse udregningen af sCT
billedet så vi ikke bruger lang tid på at lede efter knogle i luften rundt om
patienten.

\subsection{Attenuation correction}
Ved PET billeder måles photoner fra materiale man har sprøjtet ind i blodet. Ved
at se på hvor photonerne kommer fra kan man se hvor meget blod der flyder til
forskellige steder. Det er særligt brugbart til at identificere kræftknuder i
hjernen da kræftknuderne har et højere energiforbrug, og dermed får tilført
mere blod kan man se hvor de er henne.

Normalt bruges et CT billede for at korrigere for afbøjninger, men det  

\todochristoffer{Mangler at beskrive AC}

\subsection{Artifakter i hjerner}

\todo{Mangler at beskrive artifakter}

\subsection{Statistikgøjl}

\todochristoffer{Mangler at beskrive statistikken}

\todo{Overvej ordenen på næste 3 sections}

\subsection{Arbejdsgangen}

\todochristoffer{Arbejdsgangen er ikke beskrevet}

På Rigshospitalet er fremgangsmåden med MR
hjerneskanninger, at man også laver et CT scan. T1 billedet fra MR scanneren
og CT co-registreres herefter og sammenlægges til et attenuationskorrigeret
uMap kaldet FullCT. Derudover benyttes et Dixon uMap til at sætte dimensioner
på det nye uMap. Dette rekonstrueres på hospitalets scanner.


\subsubsection{FullCT}

\todolasse{FullCT er ikke beskrevet. Snakker nærmere med Claes og justerer
også i ovenstående snit}

Det er noget funk

\subsubsection{Rekonstruktion}

\todolasse{Rekonstruktion er ikke beskrevet. Skal snakke med et par folk.
Hvordan det }
