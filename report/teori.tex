\section{Teori}
\subsection{Kort introduktion til de forskellige datatyper og dannelsen af disse.}


Til metoden benyttes Magnetic resonance- (MR) og Computed
tomography-billedsekvenser (CT). CT er en røngtenscanning, som måler
vævs evne til at blokere stråling, hvilket har en direkte relation
til elektrondensitet. Knoglen er meget tydelig på disse billeder. MR
magnetiserer kroppen ganske svagt. Når dette stoppes vil kroppens
magnetfelt bruge ganske kort tid på at gendannes, og der måles på
denne for at kortlægge struktur.

Vi benytter 5 MR-sekvenser: T2, som måles efter kroppen har gendannet en vis
mændge magnetisme, og derudover 4 Ultrashort echo time MR-sekvenser (UTE)
optaget fra 2 forskellige vinkler med varierende ekko tider. Når man sænker ekkotiden betydeligt, bliver det muligt at se knogle på MR-billeder, men ikke med
en præcision, som kan matche CT.

Mens vi ikke træner på Positron Emission Tomography-billeder (PET), så
er målet med udregning af et sCT, at producere attenuationskorrigerede
PET-billeder. PET-billeder måles ved, at injicere et radioaktivt
sukkersporstof ind i en patient og måle henfaldet af dette. Da
kræftceller kræver mere energi end andre celler, vil størstedelen af
sporstoffet blive bundet til kræftcellerne, og strålingsaktiviteten
vil derfor være størst i dette område.

\subsection{Attenuation correction} 

Ved PET-billeder måles fotoner fra det radioaktive sporstof, som man
har injiceret ind i blodet. Ved at se hvor fotonerne oprinder fra, kan
man se, hvor meget blod der flyder til forskellige steder. Det er særligt
brugbart til at identificere kræftknuder i hjernen, da kræftknuderne har
et højere energiforbrug, og dermed får tilført mere af stoffet. Ligeledes er det nyttigt ved patienter med
hukommelsesproblemer, da man kan se, hvor i hjernen der er for lidt
aktivitet.

Men fotonerne går ikke igennem alle materialer lige let. Det bløde væv
inde i hjernen har en nogenlunde jævn modstand, mens kranie afbøjer
meget kraftigt, og luft næsten slet ikke gør. Da billederne bliver
optaget på en PET/MR-scanner, er det ikke muligt, at se forskel på luft
og knogle. Derfor kan man ikke korrigere for disse målingsfejl, og de
målte værdier bliver forvrængede.

For at udbedre dette benyttes der et CT billede, målt
på en anden scanner, til at korrigere. Man erstatter $\mu$-mappet
optaget på PET/MR-skanneren med CT'et og rekonstruerer PET-billedet med
korrigerede værdier.

\subsection{Artefakter i hjerner}

Artefakter i hjernen betegner mangler i MR-billederne. Ofte er de
forårsaget af implantater i tænderne, der forstyrrer magnetfelterne.
På MR-billederne optræder de som sorte områder uden data.

I forhold til vores metode er der åbenlyse problemer ved både at
træne med MR-billeder der har artefakter eller udregne sCT-billeder fra
MR-billeder med artefakter, eftersom der ikke er noget
information i disse områder.


\subsection{K-means}

K-means er en metode, der organiserer et datasæt ${x_1,...,x_N}$
bestående af $N$ punkter i et flerdimensionalt rum ind i $K$ clusters~\cite{bishop}.
Man definerer $K$ clusters med centrum $\mu_k$, og målet er da at
minimere den summerede afstand for alle datapunkter til nærmeste clustercentrum. Hvis man for hvert datapunkt definerer $K$ binære indikatorvariable $r_{nk}$, som beskriver, hvilket cluster datapunktet tilhører,
hvor $r_{nk}$ er lig 1, hvis den tilhører det $k$'te cluster, og resten
er lig 0. Da kan man definere en funktion, som beskriver den summerede
afstand som:

$$
J = \sum_{n=1}^{N} \sum_{k=1}^{K} r_{nk} || x_n - \mu_k ||^2
$$

Målet er da at minimere denne funktion ved at finde de korrekte
tilhørsforhold $r_{nk}$ og clusterplaceringer $\mu_k$.

Algoritmen startes med nogle clusterplaceringer, der typisk
er tilfældigt valgt. Derefter finder man for hvert datapunkt det
nærmeste center, og summerer den samlede afstand. For at finde den
optimale inddeling skal denne afstand minimeres. Den minimeres iterativt
ved at gentage to trin. Det første er at finde det nærmeste centrum for
hvert datapunkt, og i næste trin rykkes hvert af centrene for at minimere
den samlede afstand i forhold til deres tilhørende datapunkter. Dette
gøres blot ved at udregne gennemsnit for alle punkterne i hvert cluster. Dette
fortsættes indtil afstanden stopper med at ændre sig, hvilket er sikret,
da begge trin reducerer den summerede afstand.

Den finder dog ikke nødvendigvis det globale minimum, men blot et lokalt
minimum. Da vi blot benytter k-means til at initialisere expectation
maximization, forsøger vi ikke at håndtere dette.

\subsection{Expectation maximization}

\todolasse{Evt. omskriv EM}

Expectation maximization (EM)~\cite{bishop} er en metode til at finde de mest
sandsynlige parametre i en statistisk model, hvor variablerne er
ukendte. Metoden antager, at data består af flere multivariate gaussiske
fordelinger, og returnerer en gaussisk mikstur fordelingsmodel.

Til forskel fra k-means, som udfører en hård klassificering, hvor
hvert datapunkt tilhører en enkelt gruppe, så laver EM en blød
klassificering, hvor man finder sandsynligheden for tilhørsforhold for
hvert enkelt punkt og finder et vægtet snit over alle data punkter for at
udregne $\mu_k$'s position.

Da metoden ikke nødvendigvis finder det globale minimum, men blot et
lokalt, så startes den typisk med tilfældige værdier og køres flere
gange, og den mest sandsynlige model returneres. Alternativt kan den
initialiseres med data, for at forbedre chancen for et godt resultat,
og forhindre, at metoden skal køres flere gange. Når vi initialiserer
den med k-means, så har den allerede et bud på et cluster til alle
datapunkter, og opgaven er da blot at "blødgøre" klassificeringen
og lave gaussiske fordelinger i stedet. I expectation-trinet benyttes
den nuværende estimering af de ukendte variable til at udregne den
forventede loglikelihood funktion. I maximization-trinet findes
de parametre, som maximerer loglikelihoodfunktionen.

K-means fordelingen har også den ulempe, at den splitter halvvejs mellem
centrene, så ved clusters af ulige størrelse, vil de mindre clusters ofte
inddrage yderpunkter fra de store. Dette undgår EM, da de gaussiske
fordelinger ikke nødvendigvis har ens spredning.


\subsection{Arbejdsgangen}


\subsubsection{FullCT}

FullCT er navnet på den implementation af CT-baseret
attenuationskorrigering, der benyttes på Rigshospitalet.

Det laves ved, at man ud over at tage et PET/MR-scan også tager et CT-scan. T1-billedet fra MR scanneren og CT co-registreres herefter, og
benyttes til at udregne et attenuationskorrigerende $\mu$-map kaldet
FullCT. Derudover benyttes headerdataen fra et Dixon $\mu$-map til
at sætte dimensioner på det nye $\mu$-map. Dette rekonstrueres på
hospitalets scanner.

\subsubsection{Rekonstruktion}

For at kunne attenuationskorrigere PET-billeder har man behov for at
genindlæse al scanningsinformationen, blot med ekstra viden, f.eks.
CT information. Siemens-scanneren gemmer den uprocesserede rådata fra
skannet, så man blot kan hente sit $\mu$-map ind, og skanneren kan da
processere rådataen med viden om knogleplacering og tage højde for det,
når den gentager forsøget.

