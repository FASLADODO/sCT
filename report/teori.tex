\section{Teori}
\subsection{Kort introduktion til de forskellige type data og dannelse af disse.}

\todochristoffer{Læs}
\todolasse{Skriv om PET effekt på MR. Mere om forskel på UTE-sekvenserne}


Til metoden benyttes Magnetic resonance- (MR) og Computed
tomography-billedsekvenser (CT). CT er en røngtenscanning, som måler
evnen til at blokere stråling, hvilket har en direkte relation til
elektrondensitet. Knoglen er meget tydelig i disse billeder. MR magnetiserer
kroppen ganske svagt, når dette stoppes vil kroppens magnetisering bruge
ganske kort tid på at gendannes, og der måles på denne for at kortlægge
struktur.

Vi benytter 5 MR-sekvenser: T1, som måles efter kroppen har gendannet en vis
mændge magnetisme, og derudover 4 Ultrashort echo time MR-sekvenser (UTE)
taget fra 2 forskellige vinkler og varierende ekko tider. Når man sænker echo
tiden betydeligt bliver det muligt at se knogle på MR billeder, men ikke med
en præcision, som kan matche CT.

Mens vi ikke træner på Positron emission tomography-billeder (PET),
så er målet med udregning af et sCT i høj grad, at producerer
attenuationskorrigerede PET-billeder. PET-billeder måles ved at injektere
et radioaktivt sukkersporstof ind i en patient og måle henfaldet af dette. Da
kræftceller kræver mere energi end andre celler, vil størstedelen af
sporstoffet blive bundet til kræftcellerne, og strålingsaktivitetenvil derfor
være størst i dette område. 

\subsection{Attenuation correction} 

Ved PET billeder måles photoner fra det radioaktive sporstof, som man har
injekterer ind i blodet. Ved at se på hvor photonerne kommer fra kan man se
hvor meget blod der flyder til forskellige steder. Det er særligt brugbart
til at identificere kræftknuder i hjernen da kræftknuderne har et højere
energiforbrug, og dermed får tilført mere af stoffet, kan man måle hvor de
er henne.

Men photonerne går ikke igennem alle materialer lige let, det bløde væv
inde i hjernen har en nogenlunde jævn modstand, mens kranie afbøjer meget
mere kraftigt og luft næsten slet ikke gør. Da billederne bliver optaget
på en PET/MR scanner er det ikke muligt, at se forskel på luft og knogle,
så man kan ikke korrigere for disse målingsfejl og de målte værdier bliver
forvrænget.

For at udbedre dette benyttes der, for eksempel, et CT billede målt på en anden
scanner til at korrigere. Man erstatter $\mu$-mappet optaget på PET/MR scanneren
med ct'et og rekonstruere PET billedet med korrigerede værdier.

\todochristoffer{Mangler at beskrive AC}


\subsection{Registrering}

\todochristoffer{Mangler referencer i teori til registrering}

Billeder taget på PET/MR og PET/CT skannere kan som regelt ikke processeres
sammen grundet flere faktorer. Patienten ligger sjældent præcis på samme
måde, billederne bliver taget i forskellige opløsninger og patienten kan have
implantater der forvrænger billederne og de ligger også i forskellige image
spaces. For at klargøre billederne skal de derfor co-registreres.

Ved co-registrering forsøger man at få alle billederne til at ligge i samme
rum. I forhold til MRI billederne er co-registrering ofte trivielt. At
co-registrere et MRI og CT billede er derimod vanskeligere. Derfor har vi valgt
to forskellige metoder.


\todo{Mangler der noget her?}

I samme omgang som vi co-registrere billederne er vi interesserede også
at finde en maske. Masken skal bruges til at begrænse udregningen af sCT
billedet så vi ikke bruger lang tid på at lede efter knogle i luften rundt om
patienten.


\subsection{Artifakter i hjerner}
Artifakter i hjernen betegner mangler i MR-billederne. Ofte er de forårsaget
af implantater i tænderne der forstyrre magnetfelterne. På MR-billederne optræder
de som sorte områder uden data.

I forhold til vores metode er der åbenlyse problemer ved både at træne med MR-billeder
med artifakter eller udregne sCT-billeder fra MR-billeder med artifakter, eftersom der
simpelthen ikke er nogen data at bruge.

\todolasse{Læs om artifakter}

\subsection{k-means}

K-means er en metode, som organiserer et datasæt ${x_1,...,x_N}$ beståemde
af $N$ punkter i et flerdimensionalt rum ind i $K$ clusters. Man definerer
$K$ clusters med centrum $\mu_k$, og målet er da at minimere den summerede
afstand for alle datapunkter til nærmeste cluster centrum. Hvis man for
hvert
datapunkt definerer $K$ binære indikator variable $r_{nk}$, som beskriver
hvilket cluster datapunktet tilhører, hvor $r_{nk}$ er lig 1, hvis den
tilhører
det $k$'te cluster og resten er lig 0. Da kan man definere en funktion,
som
beskriver den summerede afstand som

$$
J = \sum_{n=1}^{N} \sum_{k=1}^{K} r_{nk} || x_n - \mu_k ||^2
$$

Målet er da at minimere denne funktion ved at finde de korrekte
tilhørsforhold
$r_{nk}$ og cluster placeringer $\mu_k$. 

Algoritmen startes med nogle nogle clusterplacering, der typisk er
tilfældigt
valgt. Derefter finder man for hvert data punkt det nærmeste center og
summerer den samlede afstand. For at finde den optimale inddeling skal
denne
afstand minimeres. Den minimeres iterativt ved at gentage to trin. Det
første
er at finde det nærmeste centrum for hvert datapunkt og i næste trin
rykkes
hvert af centrene for at minimere den samlede afstand i forhold til deres
tilhørende datapunkter, dette gøres blot ved at udregne mean for alle
punkterne i hvert cluster. Dette fortsættes indtil afstanden stopper med
at
ændre sig, hvilket er sikret, da begge trin reducerer den summerede
afstand.

Den finder dog ikke nødvendigvis det globale minimum, men blot et
lokalt minimum. Da vi blot benytter k-means til at initialisere
expectation
maximization, forsøger vi ikke at håndtere dette.

\subsection{Expectation maximization}

Expectation maximization (EM) er en metode til at finde de mest
sandsynlige
parametre i en statistiks model, hvor variablerne er ukendte. Metoden
antager
at data består af flere multivariate gaussiske fordelinger og returnerer
en
gaussisk mikstur fordelingsmodel. 

Til forskel fra k-means, som udfører en hård klassificering, hvor hvert
datapunkt tilhører en enkelt gruppe, så laver EM en blød klassificering,
hvor
man finder sandsynlighed for tilhørsforhold for hvert enkelt punkt og
finder et
vægtet snit over alle data punkter for at udregne $\mu_k$s position.

Da metoden ikke nødvendigvis finder det globale minimum, men blot et
lokalt,
så startes den typisk med tilfældige værdier og køres flere gange, og
den mest sandsynlige model returneres. Alternativt kan den initialiseres
med
data, for at forbedre chancen for et godt resultat, og forhindre at
metoden
skal køres flere gange. Når vi initialiserer den med k-means, så har den
allerede et bud på et cluster til alle datapunkter, og opgaven er da blot
at "blødgøre" klassificeringen og lave gaussiske fordelinger i stedet. I
expectation-trinet benyttes den nuværende estimering af de ukendte
variable til
at udregne den forventede loglikelihood funktion. Derefter i
maximization-trinet findes de parametre, som maximerer
loglikelihoodfunktionen.

K-means fordelingen har også den ulempe, at den splitter halvvejs mellem
centrene, så ved clusters af ulige størrelse, vil de mindre clusters ofte
inddrage yder punkter fra de store. Dette undgår EM, da de gaussiske
fordelinger, ikke nødvendigvis har ens spredning.


\subsection{Arbejdsgangen}

\todochristoffer{Arbejdsgangen er ikke beskrevet}

På Rigshospitalet er fremgangsmåden med MR
hjerneskanninger, at man også laver et CT scan. T1 billedet fra MR scanneren
og CT co-registreres herefter og sammenlægges til et attenuationskorrigeret
uMap kaldet FullCT. Derudover benyttes et Dixon uMap til at sætte dimensioner
på det nye uMap. Dette rekonstrueres på hospitalets scanner.


\subsubsection{FullCT}

\todolasse{FullCT er ikke beskrevet. Snakker nærmere med Claes og justerer
også i ovenstående snit}

Det er noget funk

\subsubsection{Rekonstruktion}

\todolasse{Rekonstruktion er ikke beskrevet. Skal snakke med et par folk.
Hvordan det }
