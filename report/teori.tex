\section{Teori}
\subsection{Kort introduktion til de forskellige type data og dannelse af disse.}

\todolasse{Introduktion til data er ikke beskrevet}

\subsection{Registrering}

\todochristoffer{Mangler referencer i teori til registrering}

Billeder taget på MRI/PET og PET/CT skannere kan som regelt ikke processeres sammen
grundet flere faktorer. Patienten ligger sjældent præcis på samme måde, billederne
bliver taget i forskellige opløsninger og patienten kan have implantater der
forvrænger billederne. For at klargøre billederne skal de derfor co-registreres.

Ved co-registrering forsøger man at få alle billederne til at ligge i samme rum.
I forhold til MRI billederne er co-registrering ofte trivielt. At co-registrere
et MRI og CT billede er derimod vanskeligere. Derfor har vi valgt to forskellige
metoder.

I samme omgang som vi co-registrere billederne er vi interesserede også at finde
en maske. Masken skal bruges til at begrænse udregningen af sCT billedet så vi
ikke bruger lang tid på at lede efter knogle i luften rundt om patienten.

\subsection{Attenuation correction}

\todochristoffer{Mangler at beskrive AC}

\subsection{Statistikgøjl}

\todochristoffer{Mangler at beskrive statistikken}

\subsection{Arbejdsgangen}

\todolasse{Arbejdsgangen er ikke beskrevet}

\subsubsection{FullCT}

\todolasse{FullCT er ikke beskrevet}

\subsubsection{Rekonstruktion}

\todolasse{Rekonstruktion er ikke beskrevet}