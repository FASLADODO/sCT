\subsubsection{k-means}

K-means er en metoden, som organiserer data i et flerdimensionelt rum ind i k
clusters. Vi definerer k clusters med centrom $\mu_k$, og målet er da at
minimere den summerede afstand for alle datapunkter til nærmeste cluster
centrum. 

Ved start af algoritmen finder man for hvert data punkt det nærmeste center
og summerer den samlede afstand. For at finde den optimale indeling skal
denne afstand minimeres. Den minimers iterativt ved at gentage to trin. Det
første er at finde det nærmeste centrum for hvert datapunkt og i næste trin
rykkes hvert af centrene for at minimere den samlede afstand i forhold til
deres tilhørende datapunkter. Dette fortsættes indtil afstanden stopper med
at ændre sig, hvilket er sikret, da begge trin reducerer den summerede
afstand. 

Den finder dog ikke nødvendigvis det globale minimum, men blot et
lokalt minimum. Da vi blot benytter k-means til at initialiserer expectation
maximization, forsøger vi ikke at håndtere dette.





