\section{T1 eller T2}

Ved MR er der to forskellige sekvenser man bruger for at lave et billede, den ene er T1 sekvensen, og den anden er T2 sekvensen. Hver sekvens har sine fordele, men oftest har Rigshospitalet brugt T1 sekvenserne til dianogsticering fordi de producere billeder i højere opløsning. Til gengæld er visse vævstyper mere fremtrædene på T2 billederne, specielt ventriklerne og væsken ved ydredelen af hjernen. Ved sCT er T2 sekvensen en fordel da den skaber større kontrast til knoglen, og formindsker misklassificering omkring ventriklerne. Derfor brugte Johannsonn et al. T2 sekvenser i deres forsøg.

Rigshospitalet har dog bedre T1 sekvenser, samt de består af flere slices og dækker derfor mere af hovedet. Det er specielt et problem, at T2 sekvensen ofte mangler den øverste del af kraniet. Derfor har der været et ønske om at teste brugen af T1 frem for T2.

Ved starten af projektet har vi fået hentet T1 sekvenserne, men ikke T2 sekvenserne, hvilket også gav anledning til håb om at vi kunne nøjes med T1.

Dog kan vi konstatere at brug af T1 sekvensen i stedet for T2 resulterer i væsentlig misklassifikation omkring ventriklerne, hjerneskallen blev for tyk, og fragmenter af knogle kunne også observeres i selve hjernemassen. 

På grund af disse observationer har vi valgt ikke at rekonstruere PET billeder ved hjælp af sCT baseret på T1 sekvensen - vi mente ikke det var den væsentlige tidsinvestering værd.

\todo{indsæt billeder og vis med cirkler hvor misklassificeringen fremkommer.}