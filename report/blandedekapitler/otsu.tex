\subsubsection{Otsu}

Omkring patienterne på de optagede billeder er der en del støj, som vi ikke
er interesserede i. For eksempel er der på alle CT billederne en hovedholder.
Dette kan forstyrre vores algoritme. For at undgå dette benytte vi et Otsu
filter (kilde) til at dele vores data op i hovede og ikke hovede. Mere konkret
bruger vi den til at udregne en maske, som vi kan benytte på alle billeder af
samme patient efter de er registrerede. 

Algoritmen opdeler billedet i to clusters baseret på deres voxel værdier. Når
man rykker threshold den ene vej eller den anden, så øger man spredningen i
det ene cluster, mens det andets bliver mindre. Ideen er da at finde det
threshold, som minimerer den kombinerede spredning. 

(Matematik)

Resultaten af metoden er et sort og hvidt billede. For at lave det om til en
maske finder vi det største sorte område, altså det rundt om hovedet. For
at sikre at masken ikke dækker dele af kraniet på andre billeder end det vi
udregner masken fra, da kræver vi at alt inde for en 3x3x3 radius også er sort. 
