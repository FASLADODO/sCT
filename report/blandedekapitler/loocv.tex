\todo{figurhenvisninger}


\todolasse{Indsæt pet billeder}

På FullCT-PET-rekonstruktionen og sCT-PET-rekonstruktionen, er der
forskelle, som øjeblikkeligt falder i øjnene. I det øverste af kraniet
går der en lodret streg med luft. Den markerer, hvor T2'en stopper
og ligeledes, går der en streg gennem bunden af lillehjernen og udenfor
hjernen ses også forskelle i næseregionen. Ved nærmere iagtelse ses
der også andre forskelle, for eksempel er lillehjernen mere udtalt på
fullct'en end på sct. Udover dette virker de meget ens.

\todolasse{indsæt forskelsbilleder fusioneret}

På forskelsbilledet ses ligeledes en stor forskel i top og bund af
kraniet. Derudover ses en stor forskel i ventriklerne i midten af
hjernen, men rundt om ventriklen ser forskellen ikke ud til at være så
stor. Procentforskellen i hjernen ser ud til at blive større jo tættere
på knoglen.

Vi har valgt at fokusere på tre regioner af hjernen, hvor udslaget ser ud
til at være størst. Forrest i hjernen omkring næsen og øjnene, bag
i hjernen øverst og lillehjernen.

\todolasse{Indsæt billeder af regionerne}

\paragraph{Forrest}

\begin{center}
    \begin{tabular}{| l | l | l | l | l |}
    \hline
    x & Gennemsnit & Median & Andel over 5 \% & Andel over 10 \% \\ \hline
    Patient 1 & 6,31 \% & 4,91 \% & 48,86 \% & 13,19 \% \\ \hline
    Patient 2 & 3,38 \% & 1,67 \% & 18,33 \% & 8,16 \% \\ \hline
    Patient 3 & 3,64 \% & 2,15 \% & 19,92 \% & 18,15 \% \\ \hline
    Patient 4 & 4,15 \% & 3,35 \% & 23,76 \% & 5,16 \% \\ \hline
    Patient 5 & 2,66 \% & 1,89 \% & 10,33 \% & 3,36 \% \\ \hline
    \end{tabular}
\end{center}

\paragraph{Bagerst}

\begin{center}
    \begin{tabular}{| l | l | l | l | l |}
    \hline
    x & Gennemsnit & Median & Andel over 5 \% & Andel over 10 \% \\ \hline
    Patient 1 & 3,99 \% & 3,73 \% & 14,98 \% & 1,1 \% \\ \hline
    Patient 2 & 1,98 \% & 1,17 \% & 6,99 \% & 3,04 \% \\ \hline
    Patient 3 & 2,28 \% & 1,28 \% & 9,42 \% & 2,89 \% \\ \hline
    Patient 4 & 4,05 \% & 3,89 \% & 17,34 \% & 1,22 \% \\ \hline
    Patient 5 & 1,84 \% & 1,45 \% & 4,11 \% & 0,65 \% \\ \hline
    \end{tabular}
\end{center}

\paragraph{Lille}

\begin{center}
    \begin{tabular}{| l | l | l | l | l |}
    \hline
    x & Gennemsnit & Median & Andel over 5 \% & Andel over 10 \% \\ \hline
    Patient 1 & 4,12 \% & 2,99 \% & 18,80 \% & 5,09 \% \\ \hline
    Patient 2 & 3,74 \% & 1,94 \% & 16,84 \% & 7,00 \% \\ \hline
    Patient 3 & 7,53 \% & 5,45 \% & 53,00 \% & 24,55 \% \\ \hline
    Patient 4 & 6,45 \% & 3,06 \% & 29,86 \% & 14,29 \% \\ \hline
    Patient 5 & 11,86 \% & 5,11 \% & 50.34 \% & 34,10 \% \\ \hline
    \end{tabular}
\end{center}

Lillehjernen er meget ustabil. Dette skyldes formentlig, at t2-sekvensen
ofte er stoppet før, den har fået nok hjerne og knogle med. Imens er der
mindre usving forest i hjernen, men der er stadig en stor andel, som
ligger over de 5 \%, og en del over 10 \%. Det bagerste af hjernen er
tilgengæld rigtig godt, men det er en stor region, og andelen, som ligger
tæt ved knoglen er forholdsvist lav.

Generelt ligner det ikke, at der er en stabil forskel i de forskellige
regioner i de rekonstruerede hjerner, men i alle tilfælde ud over to
tilfælde ved lillehjernen afviger mindst halvdelen af alle voxels med 
mindre end 5 \% forskellige.



