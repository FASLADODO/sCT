\subsection{PET-OVER TID}

\todo{figurhenvisninger}

\todolasse{Indsæt pet billeder}

På PET-billedet, der er rekonstrueret ud fra modellen trænet på tidlige
patienter, ser der ud til at være højere intensitet i
lavaktivitetsområderne. Det ses specielt uden for hjernen, men
lillehjernen er også blevet mindre. Udover dette ser det forreste af
hjernen meget ens ud, og det bagerste ligeså.

\todolasse{Indsæt forskelsbilleder fusioneret}



\paragraph{Forrest}

\begin{center}
    \begin{tabular}{| l | l | l | l | l |}
    \hline
    x & Gennemsnit & Median & Andel over 5 \% & Andel over 10 \% \\ \hline
    Patient 1 & 2,28 \% & 1,49 \% & 10,06 \% & 2,27 \% \\ \hline
    Patient 2 & 2,10 \% & 1,12 \% & 9,47 \% & 2,75 \% \\ \hline
    Patient 3 & 4,54 \% & 3,78 \% & 35,15 \% & 6,26 \% \\ \hline
    \end{tabular}
\end{center}

\paragraph{Bagerst}

\begin{center}
    \begin{tabular}{| l | l | l | l | l |}
    \hline
    x & Gennemsnit & Median & Andel over 5 \% & Andel over 10 \% \\ \hline
    Patient 1 & 1,14 \% & 0,83 \% & 1,71 \% & 0,06 \% \\ \hline
    Patient 2 & 1,57 \% & 1,15 \% & 4,23 \% & 0,50 \% \\ \hline
    Patient 3 & 2,20 \% & 1,73 \% & 6,92 \% & 1,21 \% \\ \hline
    \end{tabular}
\end{center}

\paragraph{Lille}

\begin{center}
    \begin{tabular}{| l | l | l | l | l |}
    \hline
    x & Gennemsnit & Median & Andel over 5 \% & Andel over 10 \% \\ \hline
    Patient 1 & 3,00 \% & 2,70 \% & 8,14 \% & 0,62 \% \\ \hline
    Patient 2 & 2,09 \% & 1,87 \% & 3,57 \% & 0,39 \% \\ \hline
    Patient 3 & 5,71 \% & 4,50 \% & 42,45 \% & 10,20 \% \\ \hline
    \end{tabular}
\end{center}

Patient et og to varierer meget lidt på tværs af de to modeller, de højere
forskelle ligger ved tæt ved knoglen og selv i de områder forbliver
forskllen sig for det meste under de 10 \%. Patient 3 har betydelige
problemer. Lillehjernen er betydeligt mindre med den ene model end den
anden, hvilket givet store udslag i den region. 


\subsection{Diskussion}



