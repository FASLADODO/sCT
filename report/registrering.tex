\section{Registrering}

Billeder taget på MR/PET og PET/CT skannere kan som regel ikke
processeres sammen grundet flere faktorer. Patienten ligger sjældent
præcis på samme måde, billederne bliver taget i forskellige
opløsninger og de bliver optaget i forskellige billederum. For at
klargøre billederne skal de derfor co-registreres.

Ved co-registrering forsøger man at få alle billederne til at ligge i
samme rum. I forhold til MRI billederne er co-registrering ofte trivielt.
At co-registrere et MR og CT billede er derimod vanskeligere. Derfor har
vi valgt to forskellige metoder.

I samme omgang som vi co-registrere billederne er vi interesserede i
også at finde en maske. Masken skal bruges til at begrænse udregningen
af sCT billedet så vi ikke bruger lang tid på at lede efter knogle i
luften rundt om patienten. Masken skal også fjerne evt. nakkestøtte og
hovedholdere på CT billederne.

\subsection{Co-registrering af UTE og T1 billeder}

Til co-registrering af UTE og T2 billederne har vi valgt at bruge
Insight ToolKit (ITK). Herfra benytter vi en implementation af Mattes
Mutual Information algoritme samt linær translation og interpolering.

\subsection{Co-registrering af UTE og CT}

Co-registrering af UTE og CT billeder er modsat UTE/T2 en ikke-triviel
opgave. Vi har valgt en landmark baseret løsning fra MINC's toolkit.
Vi har ikke selv skrevet denne løsning, men bruger i stedet samme
løsning som Rigshospitalet.

\subsection{Generering af maske}

Til generering af masken bruger vi først en implementering af Otsu
thresholding algoritmen for at finde en binær repræsentation. For at
sikre vores maske ikke bliver for lille udvider vi det binære billede
vha. en neighborhood-connected algoritme med 2-3 pixel i x, y og z
retningerne. Til sidst inverteres billedet hvilket efterlader os med en
binær maske.
