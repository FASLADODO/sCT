\section{Forsøgsplan}
\todo{tekst, beskriv joint histogram, omdøb "faldgrupper"}

\subsection{Leave-one-out cross validation (LOOCV)}
\subsubsection{Fremgangsmåde}
Tag 5 hjerner. Træn på 4 og test på 1 og roter.

\todo{Uddyb}

\subsubsection{Forventning}
Vi forventer, at vores joint histogram bliver nogenlunde lineær. 

\subsubsection{Faldgrupper}
Billeder med artifakter kan forstyrre algoritmen. 

\subsubsection{Resultat}
\todo{Udfør LOOCV-forsøget}

\subsection{Udregn sCT til hjerner med artifakter trænet på hjerner uden artifakter}
\subsubsection{Fremgangsmåde}
Træn på n hjerner test på m.

\subsubsection{Forventning}

\subsubsection{Faldgrupper}

\subsubsection{Resultat}

\subsection{Udregn sCT til hjerner trænet på hjerner med artifakter}
\subsubsection{Fremgangsmåde}
Iterativt træn på et antal hjerner og introducer hjerner med artifakter.
Sammenlign kvaliteten af sCT afhængigt af andelen af hjerner, som har
artifakter.

\subsubsection{Forventning}
At sCT's kvalitet forværes for hver introduceret hjerne med artifakt.

\subsubsection{Faldgrupper}

\subsubsection{Resultat}


\subsection{Over tid}
\subsubsection{Fremgangsmåde}
Træn på gamle hjerner, test på ny hjerner. Og omvendt.

\subsubsection{Forventning}
Forventningen er sCT af samme kvalitet, som dem ved LOOCV-forsøget.

\subsubsection{Faldgrupper}


\subsubsection{Resultat}

\subsection{Test på en masse}
\subsubsection{Fremgangsmåde}
Iterativt træn på n+1 hjerner

\subsubsection{Forventning}
Forventning er at efter 7-8 hjerner giver det ikke rigtig nogen
kvalitetsforskel.

\subsubsection{Faldgrupper}
Overfitting. 


\subsubsection{Resultat}

\subsection{Hvorfor 20-gaussians?}

\subsubsection{Fremgangsmåde}
Test på forskelligt antal gaussians og sammenlign resultaterne.

\subsubsection{Forventning}
Vi forventer, at kvaliteten falder under 20, men at metoden kræver mere tid
efter 20 uden betydelig forbedringer.

\subsubsection{Faldgrupper}
Overfitting. 

\subsubsection{Resultat}



