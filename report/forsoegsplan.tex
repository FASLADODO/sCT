\section{Forsøg}

Da flere af FullCT-billederne er lavet uden en perfekt maske, som skjuler hovedholder og nakkestøtten på CT-billedet, har vi valgt at generere nye FullCT, baseret på CT-billeder påført vores maske. Da støj rundt omkring hovedet påvirker attenuationkorrektionen, fik vi billeder med meget stor afvigelse imod den bagerste del af kraniet. Da vi er sikre på, at både sCT og CT billederne er påført samme maske, er der større sandsynlighed for, at vi sammenligner vores algoritme frem for støj. Der kan dog stadig fremkomme små rester af hovedstøtten tæt på kraniet, men det ser ikke ud til at være et problem.

\subsection{Validering af forsøgsresultater}

\subsubsection{Fælles histogram}

For at validere testresultaterne vil vi bl.a. benytte os af fælles histogrammer. For hvert sCT, som vi generer, har vi det korrekte CT. Ved at plotte sCT-værdier ud af y-aksen og CT værdier ud af x-aksen får vi en visuel repræsentation af ligheden mellem billederne. Hvis resultatet er identisk vil der optræde en enkelt lige linje fra $(0,0$) til $(max,max)$. Jo mere forskellige, des længere fra den lige linje ligger punkterne.

Fordelen ved denne metode er, at vi nemt kan aflæse, hvilke værdier vi har de største problemer med. Hvis vi har meget stor afvigelse i knoglerne, vil vi få en stor klump omkring 500, mens der kan være store ligheder i det bløde væv, hvilket giver en pæn linje omkring minus 500.

Derudover er det samme valideringsmetode, der bruges af Johansson et al., og vi vil derfor kunne sammenligne vores værdier med deres.

\subsubsection{Procentforskelle}

For at finde ud af om vores sCT er godt nok til attenuationskorrektion af PET-billeder, vil vi generere PET-billeder med vores sCT og det rigtige CT. Vi kan herefter konstruere et nyt billede, der er baseret på den procentvise forskel imellem FullCT-baserede PET og sCT-baserede PET. 

Ud fra procentforskelsbilledet kan vi se, hvilke regioner der er forskellige fra FullCT baserede PET-billeder, og det giver samtidigt et godt overblik over, hvor godt vores sCT-baserede PET-billede er blevet.

Begge disse valideringsmetoder er objektive, men skal evalueres manuelt. Vi forventer, at kvaliteten af sCT-billedet, og dermed også at det sCT-baserede PET-billede vil variere meget.

\subsubsection{Korrekthed}

For at bedømme korrektheden er vi nødt til at kende en parameter, som vi kan måle på. Da vi har valgt at se på procentforskelsbilleder, skal vi
derfor bestemme en procentvis afvigelse, der er godt nok til klinisk brug.
Vi har spurgt Ian Law, som er overlæge på rigshospitalet, om hvilken afvigelse
han kan acceptere, uden at det påvirker diagnosticeringen. Han mener, at
differencen i hjernen skal være under fem procent. Vi har også fundet en
artikel af Matthias Hofmann, der refererer til en grænse på ti procent~\cite{accepteretAfvigelse}. Vi
vil derfor se på billeder med en procentvis afvigelse på mindre end fem
procent som gold standard - dvs. så godt det kan blive og billeder
mellem fem og ti procent som acceptable.

\subsubsection{Regioner}

Da analyse af PET-billeder som regel er regionsbaseret, har vi valgt tre
regioner til at sammenligne. Vi har valgt lillehjernen, fordi den
typisk bliver brugt til at normalisere resten af hjernen ud fra. Derudover ved vi, at den er
et udsat område for vores modeller, da T2 ofte ikke dækker hele
lillehjernen. 

Derudover har vi valgt det forreste og det bagerste af hjernen fordi vi oplever store udslag omkring øjenhulerne og langs kanten af hjernen. Desuden er vi interesserede i, om der er den samme forskel på tværs af
hjernen, eller om den afviger mere i den ene region end den anden.

Maskerne kan ses på figur~\ref{masks}.

\begin{figure}
    \centering
   \includegraphics[width=.5\textwidth]{billeder/masks.png}
   \caption{De tre regioner vi har målt på. Den hvide er det forreste af hovedet. Den røde er den bagerste del af hovedet, den går ikke helt ud til siderne af hovedet. Den grønne er lillehjernen.}
   \label{masks}
\end{figure}

\subsection{Leave-one-out cross validation}
\subsubsection{Fremgangsmåde}

Leave-one-out cross validation (LOOCV) er en gængs metode, der bruges til
at teste korrektheden af en algoritme. I alt sin enkelthed går metoden
ud på, at vi udvælger fem patienter. Fra de fem patienter lægger vi en
patient til side, fx. den første patient, og træner algoritmen på de
resterende fire patienter. Derefter udregner vi et sCT for den første
patient. Vi gør herefter det samme igen, men udelader patient nummer to,
træner på de resterende fire og udregner nummer to's sCT.  Ved at
rotere træningssættet får vi testet metoden med data fra forskellige
patienter, og vi får en basis for at sammenligne sCT'et.

Vi har valgt dette testscenarie for at kunne sammenligne vores resultater med Johansson et al, da de benytter samme test.

\subsubsection{Forventning}
Vi forventer at vores fælles histogrammer bliver dårligere end Johansson et al.'s, da vores hidtidige sCT er visuelt dårligere end deres.


\subsubsection{Faldgrupper}
For at opnå stabile resultater er vi nødt til at filtrere patienter med artefakter fra. Da artefakter repræsenterer områder med manglende data, tror vi, at det er nødvendigt for kvaliteten af vores algoritme, at de bliver sorteret fra.

\subsubsection{Resultat}

Eftersom det primære mål med denne test er at sammenligne os med Johansson et al., har vi valgt at lave de samme figurer.

På fælleshistogrammet i figur~\ref{fig:loocv_j_h} kan vi se to koncentrationer af punkter lige omkring 0 HU. Under nul kommer vi ud i luft og blødt væv. Over 0 kan vi se, hvordan knoglen ligger. Omkring 500 findes den største spredning af punkterne.

På det kumulative diagram på figur~\ref{fig:cumm_diff_loocv} kan vi se, at vi generelt har et problem med for lave værdier i sCT i forhold til CT. Mere end 40\% af punkterne ligger for lavt.

Ud fra procentdifference (PD) billederne på figur~\ref{col:loocv_ct}  kan vi se, at algoritmen leverer nogenlunde stabile resultater. Men vi kan også se, at der er store problemer ved ventriklerne, hvor den procentvise forskel på flere af billederne sniger sig op over de ti procent.

Hvis vi ser på et PD-billede lagt over et MR-billede, er det tydeligt, at jo tættere på kraniet man ser, des større bliver afvigelsen.

Det er ikke muligt for os at evaluere korrektheden af algoritmen omkring mund- og næseregionen, da langt de fleste T2-sekvenser ikke indeholder data så langt nede. Ligeså er der problemer i toppen af kraniet, da T2-sekvenserne også mangler en skive eller to, og derfor ikke har hele hjerneskallen med.

\begin{figure}
    \centering
    \begin{subfigure}[b]{0.47\textwidth}
    	\caption{Fælleshistogram for patient 1.}
        \includegraphics[width=1\textwidth]{billeder/loocv_joint_histogram.png}
        \label{fig:loocv_j_h}
    \end{subfigure}\hfill
    \begin{subfigure}[b]{0.47\textwidth}
        \caption{Kommulativt differens diagram for alle patienter i LOOCV forsøget.}
        \includegraphics[width=1\textwidth]{billeder/cumm_diff_loocv.png}
        \label{fig:cumm_diff_loocv}
    \end{subfigure}
    \caption{På figur~\ref{fig:loocv_j_h} ses et fælleshistogram for patient 1. CT værdier er plottet ud af x-aksen og sCT værdier er plottet ud af y aksen. På figur~\ref{fig:cumm_diff_loocv} ses den kommulative afvigelse af sCT fra CT normaliseret til 1.}
    \label{fig:loocv}
\end{figure}

I gennemsnit er der lidt under ti procent afvigelse i selve hjernen.

\begin{figure}
    \centering
    \begin{subfigure}{0.3\textwidth}
        \centering
        \includegraphics[width=0.75\textwidth]{colager/loocv_pet/loocv_010476_pet_ct.png}
        \caption{FullCT for patient 1.}
        \label{col:loocv_pet_pat1_ct}
    \end{subfigure}\hfill
    \begin{subfigure}{0.3\textwidth}
        \centering
        \includegraphics[width=0.75\textwidth]{colager/loocv_pet/loocv_010476_pet_sct.png}
        \caption{PET baseret på sCT.}
        \label{col:loocv_pet_pat1_sct}
    \end{subfigure}\hfill
    \begin{subfigure}{0.3\textwidth}
        \centering
        \includegraphics[width=0.75\textwidth]{colager/loocv_pet/loocv_010476_pet_pd.png}
        \caption{Procentdifferens.}
        \label{col:loocv_pet_pat1_pd}
    \end{subfigure}\\
    \begin{subfigure}{0.3\textwidth}
        \centering
        \includegraphics[width=0.75\textwidth]{colager/loocv_pet/loocv_010769_pet_ct.png}
        \caption{FullCT for patient 2.}
        \label{col:loocv_pet_pat2_ct}
    \end{subfigure}\hfill
    \begin{subfigure}{0.3\textwidth}
        \centering
        \includegraphics[width=0.75\textwidth]{colager/loocv_pet/loocv_010769_pet_sct.png}
        \caption{PET baseret på sCT.}
        \label{col:loocv_pet_pat2_sct}
    \end{subfigure}\hfill
    \begin{subfigure}{0.3\textwidth}
        \centering
        \includegraphics[width=0.75\textwidth]{colager/loocv_pet/loocv_010769_pet_pd.png}
        \caption{Procentdifferens.}
        \label{col:loocv_pet_pat2_sub}
    \end{subfigure}\\
    \begin{subfigure}{0.3\textwidth}
        \centering
        \includegraphics[width=0.75\textwidth]{colager/loocv_pet/loocv_010850_pet_ct.png}
        \caption{FullCT for patient 3.}
        \label{col:loocv_pet_pat3_ct}
    \end{subfigure}\hfill
    \begin{subfigure}{0.3\textwidth}
        \centering
        \includegraphics[width=0.75\textwidth]{colager/loocv_pet/loocv_010850_pet_sct.png}
        \caption{PET baseret på sCT.}
        \label{col:loocv_pet_pat3_sct}
    \end{subfigure}\hfill
    \begin{subfigure}{0.3\textwidth}
        \centering
        \includegraphics[width=0.75\textwidth]{colager/loocv_pet/loocv_010850_pet_pd.png}
        \caption{Procentdifferens.}
        \label{col:loocv_pet_pat3_pd}
    \end{subfigure}\\
    \begin{subfigure}{0.3\textwidth}
        \centering
        \includegraphics[width=0.75\textwidth]{colager/loocv_pet/loocv_010960_pet_ct.png}
        \caption{FullCT for patient 4.}
        \label{col:loocv_pet_pat4_ct}
    \end{subfigure}\hfill
    \begin{subfigure}{0.3\textwidth}
        \centering
        \includegraphics[width=0.75\textwidth]{colager/loocv_pet/loocv_010960_pet_sct.png}
        \caption{PET baseret på sCT.}
        \label{col:loocv_pet_pat4_sct}
    \end{subfigure}\hfill
    \begin{subfigure}{0.3\textwidth}
        \centering
        \includegraphics[width=0.75\textwidth]{colager/loocv_pet/loocv_010960_pet_pd.png}
        \caption{Procentdifferens.}
        \label{col:loocv_pet_pat4_pd}
    \end{subfigure}\\
    \begin{subfigure}{0.3\textwidth}
        \centering
        \includegraphics[width=0.75\textwidth]{colager/loocv_pet/loocv_011030_pet_ct.png}
        \caption{FullCT for patient 5.}
        \label{col:loocv_pet_pat5_ct}
    \end{subfigure}\hfill
    \begin{subfigure}{0.3\textwidth}
        \centering
        \includegraphics[width=0.75\textwidth]{colager/loocv_pet/loocv_011030_pet_sct.png}
        \caption{PET baseret på sCT.}
        \label{col:loocv_pet_pat5_sct}
    \end{subfigure}\hfill
    \begin{subfigure}{0.3\textwidth}
        \centering
        \includegraphics[width=0.75\textwidth]{colager/loocv_pet/loocv_011030_pet_pd.png}
        \caption{Procentdifferens.}
        \label{col:loocv_pet_pat5_pd}
    \end{subfigure}
    \caption{CT, sCT og procentdifferencen for de fem patienter. Farveskalaen for procentdifferencen går fra -20 til 20. Procentdifferensbilledet er også fusioneret med T1 billedet for at man kan se hvad der er hvad.}
    \label{col:loocv_pet}
\end{figure}

På FullCT-PET-rekonstruktionerne og sCT-PET-rekonstruktionerne er der
forskelle, som øjeblikkeligt falder i øjnene. I det øverste af kraniet
går der en lodret streg med luft. Den markerer, hvor T2'en stopper,
og ligeledes går der en streg gennem bunden af lillehjernen. Udenfor
hjernen ses også forskelle i næseregionen. Ved nærmere iagttagelse ses
der også andre forskelle, for eksempel er lillehjernen mere udtalt på
FullCT'en end på sCT. Udover dette virker de meget ens.

På forskelsbillederne ses ligeledes en stor forskel i top og bund af
kraniet. Derudover ses en stor forskel i ventriklerne i midten af
hjernen, men rundt om ventriklen ser forskellen ikke ud til at være så
stor. Procentforskellen i hjernen virker til at blive større tættere
på knoglen.


\begin{figure}
    \centering
    \begin{tabular}{| l | c | c | c | c |}
        \hline
         & Gennemsnit & Median & Andel over 5 \% & Andel over 10 \% \\ \hline
        Patient 1 & 6,31 \% & 4,91 \% & 48,86 \% & 13,19 \% \\ \hline
        Patient 2 & 3,38 \% & 1,67 \% & 18,33 \% & 8,16 \% \\ \hline
        Patient 3 & 3,64 \% & 2,15 \% & 19,92 \% & 18,15 \% \\ \hline
        Patient 4 & 4,15 \% & 3,35 \% & 23,76 \% & 5,16 \% \\ \hline
        Patient 5 & 2,66 \% & 1,89 \% & 10,33 \% & 3,36 \% \\ \hline
    \end{tabular}
    \caption{Tabel for forskellen i det forreste af hjernen. Læg mærke til at andelen over 10\% for alle ligger meget højt. 10\% er øverste grænse for, hvad vi kan være tilfredse med, hvilket betyder at sCT har store problemer i det forreste af hjernen.}
    \label{tab:loocv_forresthjerne}
\end{figure}

\begin{figure}
    \centering
    \begin{tabular}{| l | c | c | c | c |}
        \hline
         & Gennemsnit & Median & Andel over 5 \% & Andel over 10 \% \\ \hline
        Patient 1 & 3,99 \% & 3,73 \% & 14,98 \% & 1,1 \% \\ \hline
        Patient 2 & 1,98 \% & 1,17 \% & 6,99 \% & 3,04 \% \\ \hline
        Patient 3 & 2,28 \% & 1,28 \% & 9,42 \% & 2,89 \% \\ \hline
        Patient 4 & 4,05 \% & 3,89 \% & 17,34 \% & 1,22 \% \\ \hline
        Patient 5 & 1,84 \% & 1,45 \% & 4,11 \% & 0,65 \% \\ \hline
    \end{tabular}
    \caption{Tabel for forskellen i det bagerste af hjernen. Den bagerste del af hjernen er langt bedre end den forreste, som set på figur~\ref{tab:loocv_forresthjerne}. Det skulle egentligt være godt, men det tyder også på, at vi har store regionale problemer med sCT, hvilket er værre, end hvis hele hjernen havde været ensformigt 10\% forkert.}
    \label{tab:loocv_bagersthjerne}
\end{figure}

\begin{figure}
    \centering
    \begin{tabular}{| l | c | c | c | c |}
        \hline
         & Gennemsnit & Median & Andel over 5 \% & Andel over 10 \% \\ \hline
        Patient 1 & 4,12 \% & 2,99 \% & 18,80 \% & 5,09 \% \\ \hline
        Patient 2 & 3,74 \% & 1,94 \% & 16,84 \% & 7,00 \% \\ \hline
        Patient 3 & 7,53 \% & 5,45 \% & 53,00 \% & 24,55 \% \\ \hline
        Patient 4 & 6,45 \% & 3,06 \% & 29,86 \% & 14,29 \% \\ \hline
        Patient 5 & 11,86 \% & 5,11 \% & 50.34 \% & 34,10 \% \\ \hline
    \end{tabular}
    \caption{Tabel for forskellen i lillehjernen. Lillehjernen fortæller meget den samme historie som figur~\ref{tab:loocv_forresthjerne}. Der er dog stor usikkerhed omkring kvaliteten af sCT specielt omkring lillehjernen. Problemet stammer fra for korte T2-sekvenser.}
    \label{tab:loocv_lillehjerne}
\end{figure}


Lillehjernen er meget ustabil. Dette skyldes formentlig, at T2-sekvensen
ofte er stoppet før, den har fået nok hjerne og knogle med. Imens er der
mindre udsving forrest i hjernen, men der er stadig en stor andel, som
ligger over de 5 \%, og en del over 10 \%. Det bagerste af hjernen er
til gengæld rigtig godt, men det er en stor region, og andelen, som ligger
nær ved knoglen, er forholdsvist lav.

Generelt ligner det ikke, at der er en stabil forskel i de forskellige
regioner i de rekonstruerede hjerner, men i alle tilfælde, ud over to
tilfælde ved lillehjernen, afviger mindst halvdelen af alle voxels med
mindre end 5 \%.


\subsection{Over tid}
\subsubsection{Fremgangsmåde}

Der har fra Rigshopitalets side været en interesse i at undersøge, om
metoden ville virke fremadrettet. Altså om en model trænet på noget data
fra én periode, ville kunne bruges til at udregne et sCT for en patient
fra en helt anden periode. 

For at undersøge dette, har vi taget de seks
tidligst optagede patienter med målte UTE-sekvenser, som ikke havde artefakter, der
var nye nok til, at den T2-serie, som vi træner udfra, er optaget.
Derudover tog vi de seks nyeste UTE-patienter, ligeledes uden artefakter.

Derefter trænede vi to modeller. En på de gamle patienter og en på de
nye. Vi lavede da sCT'er for de gamle patienter på den nye model og
omvendt. For at vurdere om det virker, vil vi undersøge sammenhængen mellem værdierne i billederne, hvis de er forskellige på en ujævn måde, vil det være et problem. Det vil betyde, at man løbende skal træne nye modeller for at få gode resultater.

\subsubsection{Forventning}

Forventningen er, at de to sCT er af samme kvalitet.
Ændringer på PET/MR-scanneren kan sagtens påvirke resultaterne af et MR-scan, men så længe ændringerne er nogenlunde lige fordelt, er det ikke
et problem, da MR-units er arbitrære, og ikke kan sammenlignes fra scan til
scan. Der skal være en region af hjernen, som bliver målt anderledes i
forhold til resten af hjernen, før der bliver problemer.

\subsubsection{Faldgrupper}

Vores skanninger forløber ikke over så lang en periode, og vi har
kun valgt at sammenligne to perioder. Så det kan for det første
bare ikke have haft nogen effekt med de hidtidige ændringer, eller
måske har der været flere ændringer indimellem, som har påvirket
skanningsresultaterne, men på seneste tidspunkt er resultaterne meget
lig de tidlige.

\begin{figure}
    \centering
    \begin{subfigure}{0.3\textwidth}
        \centering
        \includegraphics[width=0.75\textwidth]{colager/over_tid_sct/over_tid_sct_121280_early.png}
        \caption{Tidligt for patient 1.}
        \label{col:over_time_sct_pat1_early}
    \end{subfigure}\hfill
    \begin{subfigure}{0.3\textwidth}
        \centering
        \includegraphics[width=0.75\textwidth]{colager/over_tid_sct/over_tid_sct_121280_late.png}
        \caption{Sent.}
        \label{col:over_time_sct_pat1_late}
    \end{subfigure}\hfill
    \begin{subfigure}{0.3\textwidth}
        \centering
        \includegraphics[width=0.75\textwidth]{colager/over_tid_sct/over_tid_sct_121280_sub.png}
        \caption{Differens.}
        \label{col:over_time_sct_pat1_sub}
    \end{subfigure}\\
    \begin{subfigure}{0.3\textwidth}
        \centering
        \includegraphics[width=0.75\textwidth]{colager/over_tid_sct/over_tid_sct_140547_early.png}
        \caption{Tidligt for patient 2.}
        \label{col:over_time_sct_pat2_early}
    \end{subfigure}\hfill
    \begin{subfigure}{0.3\textwidth}
        \centering
        \includegraphics[width=0.75\textwidth]{colager/over_tid_sct/over_tid_sct_140547_late.png}
        \caption{Sent.}
        \label{col:over_time_sct_pat2_late}
    \end{subfigure}\hfill
    \begin{subfigure}{0.3\textwidth}
        \centering
        \includegraphics[width=0.75\textwidth]{colager/over_tid_sct/over_tid_sct_140547_sub.png}
        \caption{Differens.}
        \label{col:over_time_sct_pat2_sub}
    \end{subfigure}\\
    \begin{subfigure}{0.3\textwidth}
        \centering
        \includegraphics[width=0.75\textwidth]{colager/over_tid_sct/over_tid_sct_210445_early.png}
        \caption{Tidligt for patient 3.}
        \label{col:over_time_sct_pat3_early}
    \end{subfigure}\hfill
    \begin{subfigure}{0.3\textwidth}
        \centering
        \includegraphics[width=0.75\textwidth]{colager/over_tid_sct/over_tid_sct_210445_late.png}
        \caption{Sent.}
        \label{col:over_time_sct_pat3_late}
    \end{subfigure}\hfill
    \begin{subfigure}{0.3\textwidth}
        \centering
        \includegraphics[width=0.75\textwidth]{colager/over_tid_sct/over_tid_sct_210445_sub.png}
        \caption{Differens.}
        \label{col:over_time_sct_pat3_sub}
    \end{subfigure}
    \caption{CT og sCT for de tre patienter. På differensbilledet er det orange ingen afvigelse. Mørkere farver går mod negativ afvigelse og lysere farver går mod positiv afvigelse.}
    \label{col:over_time_sct}
\end{figure}

\begin{figure}
    \centering
    \begin{subfigure}{0.3\textwidth}
        \centering
        \includegraphics[width=0.75\textwidth]{colager/over_tid_pet/over_tid_121280_early.png}
        \caption{Tidligt for patient 1.}
        \label{col:over_time_pet_pat1_early}
    \end{subfigure}\hfill
    \begin{subfigure}{0.3\textwidth}
        \centering
        \includegraphics[width=0.75\textwidth]{colager/over_tid_pet/over_tid_121280_late.png}
        \caption{Sent.}
        \label{col:over_time_pet_pat1_late}
    \end{subfigure}\hfill
    \begin{subfigure}{0.3\textwidth}
        \centering
        \includegraphics[width=0.75\textwidth]{colager/over_tid_pet/over_tid_121280_pd.png}
        \caption{Procentdifferens.}
        \label{col:over_time_pet_pat1_pd}
    \end{subfigure}\\
    \begin{subfigure}{0.3\textwidth}
        \centering
        \includegraphics[width=0.75\textwidth]{colager/over_tid_pet/over_tid_140547_early.png}
        \caption{Tidligt for patient 2.}
        \label{col:over_time_pet_pat2_early}
    \end{subfigure}\hfill
    \begin{subfigure}{0.3\textwidth}
        \centering
        \includegraphics[width=0.75\textwidth]{colager/over_tid_pet/over_tid_140547_late.png}
        \caption{Sent.}
        \label{col:over_time_pet_pat2_late}
    \end{subfigure}\hfill
    \begin{subfigure}{0.3\textwidth}
        \centering
        \includegraphics[width=0.75\textwidth]{colager/over_tid_pet/over_tid_140547_pd.png}
        \caption{Procentdifferens.}
        \label{col:over_time_pet_pat2_pd}
    \end{subfigure}\\
    \begin{subfigure}{0.3\textwidth}
        \centering
        \includegraphics[width=0.75\textwidth]{colager/over_tid_pet/over_tid_210445_early.png}
        \caption{Tidligt for patient 3.}
        \label{col:over_time_pet_pat3_early}
    \end{subfigure}\hfill
    \begin{subfigure}{0.3\textwidth}
        \centering
        \includegraphics[width=0.75\textwidth]{colager/over_tid_pet/over_tid_210445_late.png}
        \caption{Sent.}
        \label{col:over_time_pet_pat3_late}
    \end{subfigure}\hfill
    \begin{subfigure}{0.3\textwidth}
        \centering
        \includegraphics[width=0.75\textwidth]{colager/over_tid_pet/over_tid_210445_pd.png}
        \caption{Procentdifferens.}
        \label{col:over_time_pet_pat3_pd}
    \end{subfigure}
    \caption{Tre patienter fra midten af vores tidsperiode, hvor vi har genereret sCT ud fra modellen fra de tidlige og de sene patienter. Farveskalaen for procentdifferencen går fra -20 til 20. Procentdifferensbilledet er også fusioneret med T1-billedet, for at man kan se, hvad der er hvad.}
    \label{col:over_time_pet}
\end{figure}


På PET-billedet, der er rekonstrueret ud fra modellen trænet på tidlige
patienter, ser der ud til at være 
 intensitet i
lavaktivitetsområderne. Det ses specielt uden for hjernen, men
lillehjernen er også blevet mindre. Udover dette ser det forreste af
hjernen meget ens ud, og det bagerste ligeså.

\begin{figure}
    \centering
    \begin{tabular}{| l | c | c | c | c |}
        \hline
         & Gennemsnit & Median & Andel over 5 \% & Andel over 10 \% \\ \hline
        Patient 1 & 2,28 \% & 1,49 \% & 10,06 \% & 2,27 \% \\ \hline
        Patient 2 & 2,10 \% & 1,12 \% & 9,47 \% & 2,75 \% \\ \hline
        Patient 3 & 4,54 \% & 3,78 \% & 35,15 \% & 6,26 \% \\ \hline
    \end{tabular}
    \caption{Tabel over forskellen i det forreste af hjernen mellem PET-baseret på det tidlige sCT og det sene sCT. Patient 1 og patient 2 ser begge nogenlunde ud. Desværre har patient 3 store problemer med mere end en tredjedel af værdier der er mere end 5\% forskellige.}
    \label{tab:over_tid_forresthjerne}
\end{figure}

\begin{figure}
    \centering
    \begin{tabular}{| l | c | c | c | c |}
        \hline
         & Gennemsnit & Median & Andel over 5 \% & Andel over 10 \% \\ \hline
        Patient 1 & 1,14 \% & 0,83 \% & 1,71 \% & 0,06 \% \\ \hline
        Patient 2 & 1,57 \% & 1,15 \% & 4,23 \% & 0,50 \% \\ \hline
        Patient 3 & 2,20 \% & 1,73 \% & 6,92 \% & 1,21 \% \\ \hline
    \end{tabular}
    \caption{Tabel for forskellen i det bagerste af hjernen imellem PET baseret på det tidlige sCT og det sene sCT. Her er værdierne stort set til gold standard.}
    \label{tab:over_tid_bagersthjerne}
\end{figure}

\begin{figure}
    \centering
    \begin{tabular}{| l | c | c | c | c |}
        \hline
         & Gennemsnit & Median & Andel over 5 \% & Andel over 10 \% \\ \hline
        Patient 1 & 3,00 \% & 2,70 \% & 8,14 \% & 0,62 \% \\ \hline
        Patient 2 & 2,09 \% & 1,87 \% & 3,57 \% & 0,39 \% \\ \hline
        Patient 3 & 5,71 \% & 4,50 \% & 42,45 \% & 10,20 \% \\ \hline
    \end{tabular}
    \caption{Tabel for forskellen i lillehjernen imellem PET baseret på det tidlige sCT og det sene sCT. Egentlig ser værdierne ret gode ud her. Patient 1 og 2 ligger rigtigt godt, men patient 3 har store problemer. På figur~\ref{col:over_time_pet_pat3_pd} kan vi se T2-sekvensen ikke er gået langt nok ned, derfor kan vi ikke stole på værdierne for patient 3.}
    \label{tab:over_tid_lillehjerne}
\end{figure}

Patient et og to varierer meget lidt på tværs af de to modeller, de højere
forskelle ligger tæt ved knoglen, og selv i de områder forbliver
forskellen for det meste under de 10 \%. Patient 3 har betydelige
problemer. Lillehjernen er betydeligt mindre med den ene model end den
anden, hvilket givet store udslag i den region. 
