\section{Konklusion}

Vores implementering af metoden til udregning af sCT beskrevet i Johansson
et al. lever ikke op til deres resultater. Vi får større udslag i forhold
til CT-billederne end de gør, og på PET billederne måler vi for store
forskelle i forhold til dem, som er rekonstrueret med FullCT.  Metoden vil
måske kunne bruges til hukommelsespatienter, hvor man hovedsageligt
registrerer regionale forskelle, men ikke til patienter med hjernekræft,
hvor der skal lokaliseres aktivitetsknuder i hjernen.

Over tid ser der ud til at ske en værdiforskydning i PET billederne, som
giver lavere værdi i sCT, jo nyere patienter modellen er trænet på. Det
betyder lavere attenuering, og dermed højere værdier i de rekonstruerede
PET billeder. Derfor må det anses som nødvendigt at træne nye modeller
løbende.

Vi har undervejs i projektet lavet nogle fejl, og vi har redegjort for
muligheder for at forbedre implementeringen, så bedre resultater burde
være mulige.


